%\newpage 
\begin{appendices}
\section{Asymptotic variance of usual transient estimator}
\label{app:transient_var}
%
We give here a sketch of the proof that the asymptotic variance for the transient estimator \eqref{eq:T_estimator} is indeed \eqref{eq:asym_var_transient}. As in the proof of \cref{prop:var_ts}, it suffices to understand the single-replica estimator, denoted by $\sTest$:
%
\begin{equation}
	\sTest = \frac{1}{\eta}\int_0^T R(X_t^\eta) \, dt.
\end{equation}
%
Consider the Poisson equation% \red{(discuss solvability of Poisson eq below)} %\red{good nice ref computation: gregoire ferre thesis p. 35}
%
\begin{equation}
	-\L\phi = R.
    \label{eq:poisson_asym_var}
\end{equation}
%
The invertibility of $\L$, and the bounds \eqref{eq:Linv_bound} on its inverse follow from decay estimates from~\cref{as:decay_semigroup}. This means that \eqref{eq:poisson_asym_var} is well-posed and admits a unique solution $\phi \in L^2_0(\mu)$ whenever $R\in L^2_0(\mu)$. \red{not enough to consider $\nabla \phi$; should rely on \cref{as:L_stability} or assume regularity}

Since $\E[\sTest]$ converges to a constant as $T\to+\infty$, its variance is given by
%
\begin{align}
	\frac{1}{T}\Var\paren*{\sTest} = \frac{1}{T}\paren*{\E\bkt*{\abs*{\sTest}^2} - \E\bkt*{\sTest}^2} = \frac{1}{T}\E\bkt*{\abs*{\sTest}^2} + \bigO\paren*{\frac{1}{T}}.
\end{align}
%
Applying It\^o's formula to $\phi$ allows us to write
%
\begin{equation}
	d\phi(X_t) = \L\phi(X_t) \, dt + \sigma^\t \nabla \phi(X_t) \, dW_t.
\end{equation}
%
Integrating in time from $0$ to $T$ then multiplying the result by $1/(\eta\sqrt{T})$ yields
%
\begin{equation}
	\frac{1}{\eta\sqrt{T}}\int_0^T R(X_t) \, dt = \frac{\phi(X_0) - \phi(X_t)}{\eta\sqrt{T}} + \frac{1}{\eta\sqrt{T}}\int_0^T \sigma^\t \nabla\phi(X_t) \, dW_t.
\end{equation}
%
Taking the expectation of the quantity above squared, and applying It\^o isometry to the martingale term gives
%
\begin{align}
	\frac{1}{T}\E\bkt*{\abs*{\frac{1}{\eta}\int_0^T R(X_t) \, dt}^2} = \frac{1}{\eta^2T}\int_0^T\E\bkt*{\nabla\phi(X_t)^\t \sigma\sigma^\t\nabla\phi(X_t)} \, dt + \bigO\paren*{\frac{1}{T}}.
\end{align}
%
Note that the $\bigO(1/T)$ term is indeed bounded since the resolvent $\L^{-1}$ and semigroup $\e^{t\L}$ are bounded operators in $L^2(\mu)$ (again by \cref{as:decay_semigroup}). \red{[gab says about previous sentence (rtp if need be): so what? you already said $\phi\in L^2$. Need to specify distribution of $X_0$ if one wanted to be precise (I suggest not to do so)]} By the ergodicity of the dynamics, it holds that
%
\begin{equation}
	\lim_{T\to+\infty}\frac{1}{\eta^2T}\int_0^T\E\bkt*{\nabla\phi(X_t)^\t \sigma\sigma^\t\nabla\phi(X_t)} \, dt = \frac{1}{\eta^2}\int_\mathcal{X}\nabla\phi(X_t)^\t \sigma\sigma^\t\nabla\phi(X_t) \, d\mu.
\end{equation}
%
Integration by parts then gives the desired result:
%
\begin{align}
	\lim_{T\to+\infty} \frac{\Var(\sTest)}{T} &= \frac{1}{\eta^2}\int_\mathcal{X}\sigma^\t \nabla\phi(X_t)\nabla \phi(X_t)^\t \sigma \, d\mu \\
	&= \frac{2}{\eta^2}\int_\mathcal{X} \phi(-\L\phi) \, d\mu \\
	&= \frac{2}{\eta^2}\int_\mathcal{X} R(-\L^{-1}R) \, d\mu,
\end{align}
%
which is precisely \eqref{eq:asym_var_transient}.

\begin{comment}
\section{Taylor expansions}
\red{(only the notation part will remain in the final paper; keeping it for now while making precise the details in Prop 1)}

\red{TODO: to be included where relevant in pages}

Let $\nabla^n f = \nabla \otimes (\nabla^{n-1} f)$ denote the $n$th-order derivative tensor (note that $\nabla^2 f$ denotes the Hessian).

Equivalent notations for Taylor expansion remainder:
%
\begin{itemize}
	\item Tensor:
	%
	\begin{align}
		\mathcal{R} = \frac{1}{6} \nabla^3 f\left(\Theta_\eta(x)\right)\cdot (\varphi_1(x)+\eta\varphi_2(x))^{\otimes 3}
	\end{align}
	%
	\item Pointwise:
	%
	\begin{align}
		\mathcal{R} &= \frac{1}{6}\sum_{i=1}^d\sum_{j=1}^d\sum_{m=1}^d \frac{\partial^3 f(\Theta_\eta(x))}{\partial x_i \partial x_j \partial x_k}(\varphi_1+\eta\varphi_2)_i(\varphi_1+\eta\varphi_2)_j(\varphi_1+\eta\varphi_2)_k
	\end{align}
	%
	where $(\varphi_1+\eta\varphi_2)_i = \varphi_{1,i}(x) + \eta\varphi_{2,i}(x)$.
	%
	\item Multi-index:
	%
	\begin{align}
		\mathcal{R} &= \sum_{|\alpha|=3} \frac{(\varphi_1(x)+\eta\varphi_2(x))^\alpha}{6}(\partial^\alpha f)(\Theta_\eta(x))
	\end{align}
	%
	with
	%
	\begin{gather}
	\alpha = (\alpha_1,\dotsc,\alpha_m), \qquad |\alpha| = \sum_{i=1}^m \alpha_i, \qquad \alpha_i\in\N \\
		\partial^\alpha f = \partial_1^{\alpha_1}f\cdots \partial_m^{\alpha_m}f = \frac{\partial^{|\alpha|} f}{\partial x_1^{\alpha_1}\cdots \partial x_m^{\alpha_m}}
	\end{gather}
\end{itemize}	
\end{comment}
\end{appendices}