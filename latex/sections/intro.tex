\section{Introduction}
\label{sec:introduction}
%
When considering large systems of interacting particles, quantities of interest are typically macroscopic properties, such as temperature and pressure, rather than microscopic ones. Generally, full microscopic descriptions are not only too large to be reasonably considered, but also largely uninteresting. From a numerical viewpoint, molecular dynamics provides an effective way of bridging the microscopic and macroscopic properties of such systems through computer simulations; see \cite{tuckerman2010, leimkuhler2015,allen2017} for reference textbooks. These simulations are typically done via the numerical realization of a stochastic differential equation (SDE), such as the Langevin dynamics, which evolves the positions~$q$ and momenta~$p$ as
%
\begin{equation}
\begin{aligned}
\begin{cases}
    dq_t = M^{-1} p_t \, dt, \\
    dp_t = -\nabla V(q_t) \, dt - \gamma M^{-1} p_t \, dt + \sqrt{\dfrac{2\gamma}{\beta}} \, dW_t,
    \label{eq:lang_dynamics}
\end{cases}
\end{aligned}
\end{equation}
%
where $V$ is the potential energy function, $M$ the mass matrix, $\gamma>0$ the damping coefficient, $\beta>0$ the inverse temperature and~$W_t$ a standard multidimensional Brownian motion.

One particular application of molecular dynamics is the computation of transport coefficients (such as the diffusivity, mobility and shear viscosity), which encode important physical properties of materials, and in particular measure how quickly a perturbed system returns to steady-state. At the microscopic level, transport coefficients are defined as the proportionality constant between the magnitude $\eta\ll 1$ of some external forcing exerted on the system, and some flux induced by this forcing. The flux is represented as the steady-state average~$\E_\eta(R)$ for some given observable $R$ with average 0 with respect to the equilibrium system $(\eta=0)$. This can be made precise through the framework of \emph{linear response theory}; see \cite[Chapter 8]{chandler1987} for an introduction. To numerically realize this, one considers a nonequilibrium system by adding a perturbation of magnitude $\eta$ to the reference dynamics at hand (e.g.\ Langevin dynamics), and the appropriate flux is then measured as a time-average over a long trajectory; this is known as the nonequilibrium molecular dynamics (NEMD) method.

Alternatively, the linear response can be reformulated as an equilibrium integrated correlation, known as the \emph{Green--Kubo} (GK) formula \cite{green1954,kubo1957}. Both the NEMD and Green--Kubo methods are commonly used, and each has their advantages and drawbacks; see \cite{stoltz2024} for a detailed numerical comparison of both approaches. 

Although less common, a third class of techniques consists of methods based on transient dynamics, which typically rely on monitoring the system's relaxation to steady-state after an initial perturbation (unlike the NEMD and GK methods, which are based on steady-state averages). As will be made precise in \cref{subsec:num_tech}, transient methods can be applied in two different ways: (i) starting from an equilibrium system with perturbed initial conditions, and allowing the system to relax back to its equilibrium steady-state, e.g.\ the momentum impulse relaxation \cite{arya2000} and the approach-to-equilibrium molecular dynamics methods \cite{lampin2013}; or a somewhat dual approach, carried out by (ii) applying a driving force to an equilibrium system and monitoring its relaxation towards a nonequilibrium steady-state, such as the transient-time correlation function method \cite{morriss1987,evans1988}.

All three classes of methods suffer from severe numerical difficulties, in particular with the presence of large statistical error being the main challenge, as made precise in \cref{subsec:num_tech}. For NEMD, the statistical error mainly arises from the large signal-to-noise ratio (due to the small magnitude of the perturbation $\eta$), which requires long integration times to offset the variance. For Green--Kubo, the statistical error scales linearly with the integration time $T$ due to the decay of correlations, as it amounts to integrating a small quantity plagued by a large statistical error \cite{sousaoliveira2017}.
%
There have been several attempts at more efficient methods to compute transport coefficients in the context of variance reduction \cite{pavliotis2023,spacek2023,blassel2024}. In particular, one such method is known as the \emph{subtraction technique}, developed and investigated in \cite{ciccotti1975}, and further explored in \cite{ciccotti1979}. The method is based on NEMD, and its key idea relies on estimating the equilibrium quantity $\E_0(R)$ in addition to the usual nonequilibrium response $\E_\eta(R)$. One would then subtract the estimated equilibrium trajectory (which has average 0 with respect to the stationary distribution by definition, thus leaving the transport coefficient unchanged) from the perturbed trajectory, yielding an estimator with lower variance provided that the two trajectories are sufficiently correlated.

As discussed in both works \cite{ciccotti1975,ciccotti1979} (which consider a deterministic framework), the high correlation between trajectories is a natural artifact of the deterministic dynamics for reasonably short integration times. This allows for the statistical error to be effectively subtracted out through the equilibrium trajectory, thus making the subtraction step an effective control variate. In stochastic settings, however, using independent noises for equilibrium and nonequilibrium trajectories (corresponding to $\eta=0$ and $\eta\ne 0$, respectively) results in uncorrelated trajectories. This suggests the need for constructing a sensible coupling between the two systems; otherwise, the subtraction step would essentially amount to adding two independent Gaussians, doubling the variance of the estimator at hand. 

One way to overcome this issue is to consider couplings, which have been used as a control variate to compute transport coefficients \cite{goodman2009,garnier2022}. One particular example of a common coupling strategy is synchronous coupling, which amounts to using the same noise for both dynamics. A major challenge with coupling techniques, however, is ensuring that trajectories stay coupled for long times.
This is especially problematic for systems which rely on long-time averages for convergence, e.g.\ NEMD. Typically, one hopes to obtain convergence of time-averages before trajectories decouple, but this cannot be assumed in general without additional (and often restrictive) requirements.

Synchronous coupling, for instance, typically requires conditions such as global dissipativity in order to ensure long-time couplings of the trajectories. In general, however, global dissipativity is only obtained under strong conditions. One such example is when the drift $b$ is given by $b=-\nabla V$ for some strongly convex potential $V$, which is too restrictive a requirement for actual applications in MD. This suggests that synchronous coupling is typically impractical for estimators that require long-time integration such as NEMD, as the decoupling time is much shorter than the time needed for convergence with no global dissipativity. Although in some cases, for instance at high temperatures, synchronously coupled trajectories might not decouple at all even with no global dissipativity; see \cite{monmarche2023}. A natural way to address this problem would be to construct couplings with milder conditions which guarantee long-time couplings but this remains challenging (see for instance~\cite{darshan2024}).

We adopt an alternative viewpoint: we consider methods for which convergence of an observable is feasible over short-times. In particular, we devise a transient method, consisting of an initially perturbed trajectory relaxing to equilibrium. We thus do not require long-time averages for convergence, 
which suggests that we can use synchronous coupling under weak conditions, provided that the relaxation time is smaller than the decoupling time. Indeed, even though the dynamics might start to decouple before relaxation, the total variance might be nonetheless decreased due to the control variate, as analysis will show that variance reduction is still obtainable.

\paragraph{Outline} This work is organized as follows. We discuss in \cref{sec:trans_coeff} some standard numerical methods for approximating transport coefficients, and present an approach based on integrating dynamics in the transient regime. Then, by employing the subtraction technique to the transient method, we construct in \cref{sec:transient} an improved transient subtraction estimator. We provide some error analysis on its bias and variance. We then illustrate the efficacy of our method with numerical results for several systems in \cref{sec:numerical_lang}, namely by computing the mobility for one-dimensional Langevin dynamics, and mobility and shear viscosity for a Lennard--Jones fluid. Finally, conclusions and extensions are discussed in \cref{sec:conclusion}.